\newcommand{\such}{\,.\,}
\newcommand{\vx}{\vec{x}}
\newcommand{\vy}{\vec{y}}
\newcommand{\vu}{\vec{u}}
\newcommand{\Land}{\bigwedge}
\newcommand{\Lor}{\bigvee}
\newcommand{\mbp}{\mathit{MBP}}
\newcommand{\unsat}{\textsc{unsat}}
\newcommand{\sat}{\textsc{sat}}
\newcommand{\valid}{\textsc{valid}\xspace}
\newcommand{\invalid}{\textsc{invalid}\xspace}
\newcommand{\isSat}{\textsc{isSat}\xspace}
\newcommand{\isUnSat}{\textsc{isUNSAT}\xspace}
\newcommand*\rfrac[2]{{}^{#1}\!/_{#2}}
\newcommand{\tuple}[1]{\langle #1 \rangle}       % tuple (in mathmode)

\newcommand{\aevalalgorithm}{%
\begin{algorithm2e}[tb]
\SetAlgoSkip{}
\SetKwFor{While}{forever}{do}{}
%\SetAlgoNoLine
\SetKw{KwContinue}{continue}
\KwIn{$S(\vx), \exists \vy \such T(\vx,\vy)$}
\KwOut{return value $\in$ \{\valid, \invalid\} of ${S(\vx)\!\! \implies\!\! \exists \vy \such T(\vx,\vy)}$}
\KwData{$\textsc{SmtSolver}$, counter $i$, models $\{m_i\}$, MBPs $\{T_{i}(\vx)\}$, conditions $\{\phi_i({\vx,\vy})\}$}
\BlankLine
$\textsc{SmtAdd}(S(\vx))$; \\
$i \gets 0$; \\
\While{}{
$i$++; \\
%$res \leftarrow \textsc{SmtSolve}()$\label{alg:check_unsat_s}; \\
\lIf(\label{alg:returnUnsat}){$(\isUnSat(\textsc{SmtSolve}()))$}{\Return \valid }
$\textsc{SmtPush}()$; \\
$\textsc{SmtAdd}(T(\vx,\vy))$; \\
%$res \gets \textsc{SmtSolve}()$\label{alg:find_matching_ass};\\
\lIf(\label{alg:returnSat}){$(\isUnSat(\textsc{SmtSolve}()))$}{\Return \invalid }
$m_i \gets \textsc{SmtGetModel}()$\label{alg:model};\\ 
%$E \leftarrow extrapolate(m)$;\\
$(T_{i},\phi_i({\vx,\vy}))\! \gets\! \textsc{GetMBP}(\vy, m_i, T(\vx,\vy)))$\label{alg:proj};\\
$\textsc{SmtPop}()$;\\
$\textsc{SmtAdd}(\neg {T_{i}})$\label{alg:block}; \\
}
%\Return $res$;
\caption{\aeval \Big($S(\vx), \exists \vy \such T(\vx,\vy)$\Big)}
\label{alg:ae_val}
\end{algorithm2e}
}

\newcommand{\localfactoralg}{%
\begin{algorithm2e}[t!]
\SetAlgoSkip{}
\SetInd{0.4em}{0.4em}
\SetKwFor{ForAll}{forall}{do}{}
\SetKwFor{For}{for}{do}{}
\KwIn{$y_j \in \vy$, local Skolem relation \newline $\phi(\vx,\vy) = \Land_{y_j \in \vy}(\psi_{y_j}(\vx,y_{j},\ldots, y_{n}))$}
\KwOut{factor of the local Skolem refinement $y_j = f_j(\vx)$}
\KwData{known functions $f_{j+1}(\vx),\ldots,f_{n}(\vx)$}
\BlankLine
\BlankLine
$\pi_{y_j}(\vx,y_{j}) 
%  [  \equiv \psi_{y_j}(\vx,y_{i}, g_1(\vx),\ldots g_n(\vx)) ]
  \gets
  \textsc{Substitute}(\psi_{y_j}(\vx,y_{j},y_{j+1},\ldots,y_{n}), 
  f_{j+1}(\vx),\ldots,f_{n}(\vx))$\label{alg:loc_subst};\\
\lIf{$(\pi_{y_j}(\vx,y_{j}) \in \varnothing)$}{\Return $\varnothing$\label{alg:loc_ret_none}}
$\Land_k \big(y_{j} \sim f^k_j(\vx)\big) \gets \textsc{Rewrite}(\pi_{y_j}(\vx,y_{j}))$\label{alg:loc_rewrite};\\
\Return $\textsc{MinimalSkolem}\Big(\Land_k \big(y_{j} \sim f^k_j(\vx)\big)\Big)$\label{alg:loc_inst};\\
\caption{$\textsc{LocalFactor}(y_j, \phi(\vx,\vy)$)}
\label{alg:loc}
\end{algorithm2e}
}

\section{Witnessing existential quantifiers with \aeval}

Quantifier elimination is a decision procedure that turns a quantified formula into an equivalent quantifier-free formula.
In addition, the quantifier elimination algorithms are often able to discover a Skolem function that represents witnesses for the existentially quantified individual variables (e.g.,~\cite{DBLP:conf/cav/BalabanovJ11,DBLP:journals/sttt/KuncakMPS13,KLXJOIA,Chakraborty15}).
%
Various tasks in verification and synthesis~\cite{DBLP:conf/asplos/Solar-LezamaTBSS06,DBLP:conf/fmcad/CimattiGMT13,DBLP:conf/popl/BeyeneCPR14,DBLP:conf/nfm/GasconT14} rely on efficient techniques to remove existential quantifiers from formulas in first-order logic, thus adjusting the task to be decided by an SMT solver.
In particular, \emph{functional synthesis} aims at computing a function that meets a given input/output relation.
A function with an input $x$ and an output $y$, specified by a relation $f(x,y)$, can be constructed as a by-product of deciding validity of the formula $\forall x \exists y \such f(x,y)$.
Due to a well-known \emph{AE-paradigm} (also referred to as \emph{Skolem paradigm}~\cite{DBLP:conf/popl/PnueliR89}),
the formula $\forall x \exists y \such f(x,y)$ is equivalent to the formula $\exists \mathit{sk} \; \forall x \such f(x, \mathit{sk}(x))$, which means existence of a Skolem function $\mathit{sk}$, such that $f(x,\mathit{sk}(x))$ holds for every $x$.
Thus the key feature in  modern quantifier elimination approaches is their ability to produce witnessing Skolem function.


\subsection{Model-Based Projection for Linear Rational Arithmetic}
\label{sec:mbp}

Quantifier elimination of a formula $\exists \vy \such T(\vx,\vy)$ is an expensive procedure that typically proceeds by enumerating all models of an extended formula $T(\vx,\vy)$.
However, in some applications, the quantifier-free formula, fully equivalent to $\exists \vy \such T(\vx,\vy)$, is not even needed.
Instead, it is enough to operate by (possibly incomplete) sets of models.
This idea relies on some notion of projection that under-approximates existential quantification.
In this section, we consider a concept of Model-Based Projections (MBP), recently proposed by~\cite{komuravelli2014smt,Dutertre}.

In the following, we use vector notation to denote sets of variables (and set-theoretic operators of \emph{subset} $\vu \subseteq \vx$, \emph{complement} $\vx_{\vu} = \vx \setminus \vu$, \emph{union} $\vx = \vu \cup \vx_{\vu}$).
\begin{definition}
\label{def:mbp}
An $\mathit{MBP}_{\vy}$ is a function from models of
$T(\vx,\vy)$ to $\vy$-free formulas
iff:
\begin{gather}
\text{if }m\models T(\vx,\vy) \text{ then } m\models \mathit{MBP}_{\vy}(m,T) \label{mbp.cond1}\\
\mathit{MBP}_{\vy}(m,T) \!\implies\! \exists \vy \such T(\vx,\vy) \label{mbp.cond2}
\end{gather}
\end{definition}

There are finitely many MBPs for fixed  $\vy$ and $T$ and different models $m_1,\ldots,m_n$ (for some $n$):
$T_{1}(\vx),  \ldots, T_{n}(\vx)$, such that
$\exists \vy \such T(\vx,\vy) = \Lor_{i=1}^{n} T_{i}(\vx)$. 

A possible way of implementing an MBP-algorithm was proposed in~\cite{komuravelli2014smt}.
It is based on Loos-Weispfenning (LW)  quantifier-elimination method~\cite{loos1993applying} for Linear Rational Arithmetic (LRA).
Consider formula $\exists \vy \such T(\vx,\vy)$, where $T$ is quantifier-free.
In our simplified presentation,
$\vy$ is singleton, $T$ is in Negation Normal Form (that allows the operator $\neg$ to be applied only to variables), and $y$ appears in the literals only of the form ${y=e}$, ${l<y}$ or ${y<u}$, where $l,u,e$ are $y$-free.
LW states that the equation~\eqref{eq:formula_rewrite} holds:

\begin{equation}
  \exists y \such T (\vx) \equiv \Big( \Lor_{(y = e) \in \mathit{lits}(T)}{T[e]} \lor
  	\Lor_{(l < y) \in \mathit{lits} (T)}{T [l + \epsilon]} \lor
	T[-\infty] \Big)\label{eq:formula_rewrite}
\end{equation} 
\smallskip  

In~\eqref{eq:formula_rewrite}, $\mathit{lits}(T)$ denote the set of literals of $T$, $T[\cdot]$ stands for a \emph{virtual substitution} for the literals containing $y$.
In particular, $T[e]$ substitutes exact values of $y$ ($y=e$), $T[l+\epsilon]$ substitutes the intervals ($l < y$) of possible values of $y$, $T[-\infty]$ substitutes the rest of the literals.
Consequently, a function $\mathit{LRAProj_{T}}$ is an implementation of the $\mathit{MBP}$ function for~\eqref{eq:formula_rewrite}:

\begin{equation}
\begin{aligned}
\label{case:proj_define}
&\mathit{LRAProj}_{T}(m) = \left\{
\begin{array}{ll}
T[e], 			& \mbox{if}\ (y=e) \in \mathit{lits}(T) \land 
			m \models (y = e)\\
T[l+ \epsilon],	& \mbox{else if}\ (l < y) \in \mathit{lits}(T) 
			\land m \models (l < y) \land \\
			& \forall (l'\!<\!y)\!\in\!\mathit{lits}(T) \such \!m\!\models\!\big((l'\!<\!y)\!\!\implies\!\!(l'\! \le\! l)\big)\\
T[-\infty], 		& \mbox{otherwise}	
\end{array}
\right.
\end{aligned}
\end{equation}





\subsection{Deciding Validity of $\forall\exists$-Formulas}
\label{sim:check}

An algorithm \aeval for deciding validity of $\forall\exists$-formulas and constructing witnessing Skolem relations was presented in~\cite{fedyukovich2015automated}.
  Without loss of generality, we restrict the input formula to have the form 
${S(\vx)\implies\exists \vy \such T(\vx,\vy)}$, where
$S$ has no universal quantifiers, and $T$ is quantifier-free.


\aeval is an extension of the MBP-algorithm in~\cite{komuravelli2014smt}.
It assumes that for each projection $T_{i}$ there exists a condition $\phi_i$ under which  $T$ is equisatisfiable with $T_{i}$:
%$\phi_i(\vx,\vy)$ such that~\eqref{eq:phi} holds.
\begin{equation}
\label{eq:phi}
\phi_i(\vx,\vy) \implies \big( T_{i}(\vx) \iff T(\vx,\vy) \big)
\end{equation}
Intuitively, each $\phi_i$ captures the substitutions made in $T$ to produce $T_{i}$.
%(is a condition under which $T$ is equisatisfiable with $T_{i}$).
We assume that each $\phi_i$ is in the Cartesian form, i.e., a conjunction of
terms, in which each $y_j \in \vy$ appears at most once:
\begin{equation}
\label{eq:phi_conj}
\phi_i(\vx,\vy) = \Land_{y_j \in \vy}(\psi_{y_j}(\vx,y_{j},\ldots,y_{n}))
\end{equation}

      \aevalalgorithm  

\grigory{to re-iterate}

We write $(T_{i}, \phi_i) \gets
\textsc{GetMBP} (\vy, m_i, T(\vx, \vy))$ for the invocation of the MBP-algorithm that takes
a formula $T$, a model $m_i$ of $T$ and a vector of variables $\vy$,
and returns a projection $T_{i}$ of $T$ based on $m_i$ and the corresponding
relation $\phi_i$.


\aeval is shown in Alg.~\ref{alg:ae_val}.  Given formulas
$S(\vx)$ and ${\exists \vy \such T(\vx, \vy)}$, it
decides validity of ${S(\vx)\! \implies\! \exists \vy \!\such\!
  T(\vx, \vy)}$.  \aeval enumerates the
models of $S \land T$ and blocks them from $S$.
In each iteration $i$, it first checks whether
$S$ is non-empty (line~3) and then looks for a model $m_i$ of $S
\land T$ (line~\ref{alg:model}).  If $m_i$ is found, \aeval
gets a projection $T_i$ of $T$ based on $m_i$ (line~\ref{alg:proj})
and blocks all models contained in $T_i$ from $S$
(line~\ref{alg:block}).  The algorithm iterates until either it
finds a model of $S$ that can not be extended to a
model of $T$ (line~\ref{alg:returnSat}), or all models of $S$
are blocked (line~\ref{alg:returnUnsat}). In the first case, the input
formula is invalid. In the second case, every model of $S$ has been extended
to some model of $T$, and the  formula is valid. 

\newcommand{\skolemcases}{%
\begin{equation}
\label{case:skolem}
\mathit{Sk}_{\vy} (\vx, \vy) \equiv
\begin{cases}
  \phi_{1} (\vx, \vy)  & \text{if } T_1 (\vx) \\
    \phi_{2} (\vx, \vy)  & \text{else if } T_2 (\vx)\\
  \cdots &\text{\qquad else }\cdots \\
  \phi_{n} (\vx, \vy) & \text{\qquad\qquad else } T_n (\vx) \\
\end{cases}
\end{equation}
}

\textsc{AE-VAL} is designed to
construct a Skolem relation
${\mathit{Sk}_{\vy} (\vx, \vy)}$, that maps each
model of $S(\vx)$ to a corresponding model of $T(\vx,\vy)$.
We use a set of
projections $\{T_{i}(\vx)\}$ for $T(\vx,\vy)$ and a set of conditions $\{\phi_i
(\vx, \vy)\}$
that make the corresponding projections equisatisfiable with $T(\vx,\vy)$.
Intuitively, $\phi_i$ maps each model of $S \land T_{i}$ to a model of $T$.
Thus, in order to define the guarded Skolem relation $\mathit{Sk}_{\vy}(\vx, \vy)$ it is enough to 
match each $\phi_i$ against the corresponding $T_{i}$:

\skolemcases


\subsection{Towards Minimal Skolem Refinement}
\label{sec:new}

\localfactoralg

By construction, each local Skolem relation $\phi(\vx,\vy)$ has a form $\Land_{y_j \in \vy}(\psi_{y_j}(\vx,y_{j},\ldots,y_{n}))$.
Since quantifier elimination in \aeval is applied iteratively for each variable $y_j \in \vy$, $y_j$ may depend on the variables of $y_{j+1},\ldots, y_{n}$ that are still not eliminated in the current iteration $j$.
Each $\psi_{y_j}(\vx,y_{j},\ldots,y_{n})$ is the conjunction
$\psi_{y_j}(\vx,y_{j},\ldots,y_{n}) = \Land_{i}(cl_i(\vx,y_{j},\ldots,y_{n}))$,
where each $cl_i$ is  an (in)equality.

For each $y_j\in \vy$, our goal is to find a Skolem function $f_{y_j}(\vx)$, such that $(y_j = f_{y_j}(\vx)) \!\implies\! \exists y_{j+1},\ldots,y_{n} \such \psi_{y_j}(\vx,y_{j},\ldots,y_{n}) $.
The idea is presented in Alg.~\ref{alg:loc}.
The algorithm is applied separately for each $y_j \in \vy$, starting from $y_n$ till $y_1$.
For each $y_j$, assume, we already established Skolem functions $f_{j+1}(\vx),\ldots,f_{n}(\vx)$ for variables $y_{j+1},\ldots,y_n$ in the previous runs of the algorithm. 

First, the algorithm substitutes each appearance of variables $y_{j+1},\ldots, y_{n}$ in $\psi_{y_j}$ by $f_{j+1}(\vx),\ldots,f_{n}(\vx)$ (line~\ref{alg:loc_subst}).
If for some variable there is no Skolem function to substitute, the algorithm halts with nothing
(line~\ref{alg:loc_ret_none}).
Second, the algorithm normalizes $\pi_{y_j}(\vx,y_{j})$ into the form
$\Land_k \big(y_{j} \sim f_k(\vx)\big)$, i.e., conjunction of
expressions, left-hand-sides of which are reserved for $y_j$ and ${\sim
\in \{<, \le, =, \ge, >\}}$.
For this, it uses the method \textsc{Rewrite} (line~\ref{alg:loc_rewrite}) that rewrites each clause using the following rule (where $g,h$ - are functions over $\vx$, $p,q$ - rational numbers, $\mathit{sgn}$ - a function, returning the sign of the rational number):

\begin{equation}
{\Big( (g(\vx)+ p\!\times\!y_j}) \sim ( {h(\vx) + q\!\times\!y_j})\Big) \!\implies\! \Big(\big(\mathit{sgn}(p-q)\times y_j \big) \sim \big( -\frac{g(\vx)}{|p-q|} + \frac{h(\vx)}{|p-q|} \big)\Big)  \notag
\end{equation}
Finally the algorithm gets rid of inequalities.
%If there are exist some, $f(\vx)$ defines some interval for the values for $y_j$.
Method \textsc{MinimalSkolem} rewrites each clause using the following rules:
\begin{align}
     y_j \le g (\vx) &\!\implies\! y_j = g (\vx)  \notag \\
     y_j \ge g (\vx) &\!\implies\! y_j = g (\vx)  \notag \\
     y_j < g (\vx) &\!\implies\! y_j = g (\vx) - 1 \notag \\
     y_j > g (\vx) &\!\implies\! y_j = g (\vx) + 1  \notag \\
     y_j > g_1 (\vx) \land y_j > g_2 (\vx) &\!\implies\! y_j = max(g_1 (\vx), g_2(\vx)) + 1  \notag \\
     y_j < g_1 (\vx) \land y_j < g_2 (\vx) &\!\implies\! y_j = min(g_1 (\vx), g_2(\vx)) - 1  \notag \\
     y_j > g_1 (\vx) \land y_j < g_2 (\vx) &\!\implies\! y_j = \frac{g_2 (\vx) - g_1(\vx)} {2}  \notag \\
     ...
\end{align}



