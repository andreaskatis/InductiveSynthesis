\section{Implementation}
\label{sec:impl}

We develop \jkindsynt, our synthesis algorithm on top of \jkind~\cite{jkind},
a Java-implementation of the \textsc{KIND} model
checker~\cite{Hagen08:kind}. Each model is described using the Lustre
Specification language, which is used as an intermediate language to formally verify contracts in the
Assume-Guarantee Reasoning (\textsc{AGREE}) framework~\cite{NFM2012:CoGaMiWhLaLu}.
Internally, \jkind is using two parallel engines (for \textit{BaseCheck(n) and
\textit{ExtendCheck}}) in order to construct a k-inductive proof for the
property of interest.
The first order formulas that are being constructed are then fed to the \textsc{Z3} SMT
solver~\cite{DeMoura08:z3} which provides state of the art support for reasoning
over quantifiers and incremental search.

Provided with the assumption that the check queries are satisfiable according to
the results received by the SMT solver, we proceed to construct a collection of
Skolem functions using the \aeval Skolemizer%
\footnote{More info about \aeval can be found at~\url{http://www.inf.usi.ch/phd/fedyukovich/niagara}.}.
\aeval  uses LRA as a background logic, and thus casts all numeric variables to Reals and provides the Skolem relation over Reals as well.%
\footnote{To increase precision, of the realizability checks over LIA, \jkind has an option to use \textsc{Z3} directly.}
In future work, we plan to enhance  \aeval for Linear Integer Arithmetic (LIA) to soundly support Skolem relation over Integers.

The final step of our implementation involved the creation of a specific purpose
translation tool, which we currently call \textsc{SMTLib2C}%
\footnote{The source code
is available at \url{https://github.com/andrewkatis/SMTLib2C}}. The collection
of the Skolem functions is given as an input to the compiler, which in turn uses
them to generate implementations in the C language. We intend to further improve
the compiler's efficiency as part of an individual future work.


