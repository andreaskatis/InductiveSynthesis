\section{Implementation}
\label{sec:impl}

% \grigory{moved the paragraph from section 2 here. More info about the implementation would be desirable. Some details about smtlib2 -> C conversion, maybe}

The synthesis algorithm presented in this work is currently a feature provided
by JKind~\cite{jkind}, a re-implementation of the KIND model
checker~\cite{Hagen08:kind} in Java. Each model is described using the Lustre
Specification language, which is used as an intermediate language to formally verify contracts in the
Assume-Guarantee Reasoning (AGREE) framework~\cite{NFM2012:CoGaMiWhLaLu}.
Internally, JKind is using two parallel engines in order to construct a
k-inductive proof for the property of interest. The first order formulas that
are being constructed are then feeded to the Z3 SMT
solver~\cite{DeMoura08:z3} which provides state of the art support for reasoning
over quantifiers and incremental search.

Provided with the assumption that the check queries are satisfiable according to
the results received by the SMT solver, we proceed to construct a collection of
Skolem functions using the \aeval skolemizer. \andreas{Add \aeval related
implementation details here.}

The final step of our implementation involved the creation of a specific purpose
translation tool, which is currently called SMTLib2C~\footnote{The source code
is available at https://github.com/andrewkatis/SMTLib2C}. The translation tool
is still at a very primitive state in terms of optimizations, a goal that we
eventually want to address as part of our future work.

