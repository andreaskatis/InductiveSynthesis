\section{Future Work}
\label{sec:futurework}

The meaningful results of our work so far on the synthesis from Assume-Guarantee
contracts have also provided a solid ground towards extending and improving the
involved algorithms in the future. A particularly important milestone is to
eventually switch to a more efficient algorithm, where we endorse the core
idea of Property Directed Reachability~\cite{een2011efficient,bradley11},
using efficient ways to further enhance the algorithm's performance through the
use of implicit abstractions~\cite{cimatti2014ic3} to further reduce the search
space of the algorithm. This will also help our original work on realizability
checking, by improving the unsoundness of our unrealizable results. Another
promising idea here is the use of Inductive Validity Cores (IVCs)~\cite{Ghass16}, whose main purpose is to effectively pinpoint the absolutely necessary model elements in a generated proof. We can potentially use
the information provided by IVCs as a preprocessing tool to reduce the size of
the original specification, and hopefully the complexity of the realizability
proof. Of course, a few optimizations can be further implemented in terms of
\aeval's specific support on proofs of realizability and finally, a very
important subject is the further improvement of the compiler that we created to
translate the Skolem functions into C implementations, by introducing
optimizations like common subexpression elimination.

Another important goal is that of supporting additional theories, and primarily
LIA, which is currently not fully supported by \aeval's model based projection
technique. Finally, another potential optimization that could effectively reduce 
the algorithm's complexity is the further simplification of the transition
relation that we are currently using, by reducing its complicated form through the
mapping of common subexpressions on different conditional
branches. This will also have a direct impact on the skolem relations retrieved
by \aeval, reducing their individual size and improving, thus, the final
implementation in terms of readability as well as its usability as an
intermediate representation to the preferred target language.