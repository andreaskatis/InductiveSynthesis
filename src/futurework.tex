\section{Future Work}
\label{sec:futurework}

\grigory{this paragraph is moved from the experiments section.}
A particularly interesting way to improve upon this is by switching to a more
sophisticated algorithm, where we endorse the core idea of Property Directed
Reachability in terms of finding a proof of realizability, in conjunction with Skolem functions.

\grigory{this paragraph as well}
For future work, we hope to tackle such cases on three different frontiers. The
first is again the use of a better algorithm that can effectively reduce the size of
the transition relation used during the realizability checking algorithm.
Another interesting idea here is the use of Inductive Validity Cores
(IVCs)~\cite{Ghass16}, whose main purpose is to effectively pinpoint the
absolutely necessary model elements in a generated proof. We can potentially use the
information provided by IVCs as a preprocessing tool to reduce the size of
the original specification, and hopefully the complexity of the realizability
proof. Of course, a few optimizations can be further implemented in terms of
\aeval's specific support on proofs of realizability and finally, a very
important subject is the further improvement of the compiler that we created to
translate the Skolem functions into C implementations, by introducing
optimizations like common subexpression elimination.


While our current approach to program synthesis has been shown to be effective
in the contracts that we have exercised, there are yet a lot of interesting ways
to extend and optimize the underlying algorithm, to yield better results in the
future. An important extension is that of supporting additional theories such as
integers, which is currently not supported by \aeval's model based projection
technique. To combat the lack of soundness on unrealizable results, and thus
missing potential synthesized implementations, we will be developing a new
algorithm that is mainly based on the idea of generating inductive invariants,
much like the way that is presented in Property Directed Reachability
algorithms. Finally, another potential optimization that could effectively
reduce the algorithm's complexity is the further simplification of the
transition relation that we are currently using, by reducing its complicated
form through the mapping of common subexpressions on different conditional
branches. This will also have a direct impact on the skolem relations retrieved
by \aeval, reducing their individual size and improving, thus, the final
implementation in terms of readability as well as its usability as an
intermediate representation to the preferred target language.

\grigory{Altogether, the section is too long. It is not good for the paper to have many TODO items. Otherwise, the reviewers will think that the paper is to raw, and the TODO items should be addressed before the submission.}