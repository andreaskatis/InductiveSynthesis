\newcommand{\toolname}{\textsc{RealSynth}}

\mike{What should tool name be?  KindSyn, RealSyn, jsyn ``jason''.  I have defined it by command to make it easy to replace.}

\section{Introduction}
Automated synthesis research is concerned with discovering efficient algorithms to construct candidate programs that are guaranteed to comply with predefined temporal specifications.  This problem has been well studied for propositional specifications, especially for (subsets of) LTL~\cite{gulwani2010dimensions}.  More recently, the problem of synthesizing programs for richer theories has been examined, including work in {\em template synthesis}~\cite{srivastava2013template}, which attempts to find programs that match a certain shape (the template), and {\em inductive synthesis}, which attempts to use counterexample-based refinement to solve synthesis problems~\cite{flener2001inductive}.  Such techniques have been widely used for stateless formulas over arithmetic domains~\cite{reynoldscounterexample}.
\textit{Functional synthesis} has also been effectively used to synthesize
subcomponents of already existing partial
implementations~\cite{kuncak2010complete,kuncak2013functional}. 

In this paper,
we propose a new approach that can synthesize programs for arbitrary {\em
assume/guarantee contracts} that do not have to conform to specific template
shapes or temporal restrictions. The contracts are described using
safety properties involving real arithmetic.  Although the
technique is not guaranteed to succeed or terminate, we have used it to successfully synthesize a range of programs over non-trivial contracts.
It is more general than previous approaches for temporal synthesis involving theories, supporting both arbitrary safety properties rather than ``stateless'' properties (unlike~\cite{reynoldscounterexample}) and arbitrary shapes for synthesized programs (unlike~\cite{srivastava2013template}).

Our approach is built on previous work determining the {\em realizability} of contracts involving infinite theories such as linear integer/real arithmetic and/or uninterpreted functions~\cite{Katis15:Realizability,katis2015machine}.  The algorithm, explained in Section~\ref{sec:synthesis}, uses a quantified variant of k-induction that can be checked by any SMT solver that supports quantification.  Notionally, it checks whether a sequence of states satisfy the contract of depth $k$ is sufficient to guarantee the existence of a successor state that satisfies the contract for an arbitrary input.  An outer loop of the algorithm increases $k$ until either a solution or counterexample is found.

The step from realizability to synthesis involves moving from the existence of a witness (as can be provided by an SMT solver such as Z3 or CVC4) to the witness itself.  For this, the most important obstacle is the (in)ability of the SMT solver to handle higher-order quantification. Fortunately, interesting directions to solving this problem have already surfaced, either by extending an SMT solver with native synthesis capabilities~\cite{reynoldscounterexample}, or by providing external algorithms that reduce the problem by efficient quantifier elimination methods~\cite{fedyukovichae}.  Our synthesis relies on our previous implementation for realizability checking and the skolemization procedure implemented in the \aeval tool~\cite{fedyukovichae}.


We combined the above ideas to create a reasonable sequential synthesis
approach, which we call \toolname.  It applies the realizability checker from~\cite{Katis15:Realizability} and then extracts a Skolem witness formula from the \aeval tool that can immediately be turned into a C program.  In order to support synthesis, several changes were required to the quantifier-elimination approach to produce Skolem {\em functions} rather than {\em relations}.  The main contributions of the work are therefore:
\begin{itemize}
	\item To the best of our knowledge, the first synthesis procedure from
	Assume-Guarantee contracts, modulo infinite theories, usable in a broader
	list of applications when compared to already existing approaches.
	\item A framework of extracting fine-grained Skolem functions for $\forall\exists$-formulas in Linear Real Arithmetic
	\item A prototype tool implementing the algorithm
	\item An experiment demonstrating the application of the tool on various benchmark examples
\end{itemize}

\noindent This paper presents the first full exposition of the idea, which was originally proposed in a workshop paper without an implementation or experiment~\cite{katis2016towards}.

In Section~\ref{sec:synthesis}, we provide the necessary background
definitions that are used in our synthesis algorithm, as well as an informal
proof of the algorithm's correctness. Section~\ref{sec:aeval} contains the
core formal notions on which the \aeval Skolemizer is based, as well
as the adjustments that were done for it to better support the needs of this
work. Section~\ref{sec:impl} provides detailed
source information for each one of the important
components of this work. Section~\ref{sec:experiment} presents our results on
using the algorithm to automatically generate leaf-level component implementations for different case studies.
In Section~\ref{sec:related} we give a brief historical background on the
related research work on synthesis, and we discuss potential future work in
Section~\ref{sec:futurework}. Finally, we conclude this paper in
Section~\ref{sec:conclusion}.

\iffalse



\grigory{The intro needs work. What's the main challenge? Why do we start with formal verification, and with synthesis or realizability checking?}
Formal verification is a well-established area of research, with ever increasing
popularity, in an attempt to provide better tools to software engineers during
the design and testing phases of a project and effectively reduce its overall
development cost. A particularly interesting problem in formal verification \grigory{this is not 100\% right, since program synthesis spreads far beyond program verification. There are many applications of synthesis that are not based on verification.} is
that of program synthesis aiming to construct efficient algorithms that can
automatically generate code which is guaranteed to behave correctly based on the
information provided by the user through formal or informal requirements.
\grigory{informal requirements? how come?}

Program synthesis has been applied in a significant amount of
diverse contexts, \grigory{citations?}, but to the best of our knowledge there is no application yet that ....%
 \grigory{what's the best novelty in our synthesis application?}
In this paper, we focus on the automated
generation of implementations for the leaf-level components of embedded
systems, using safety properties that are expressed in the form of an
Assume-Guarantee contract.
In recent work~\cite{katis2016towards}, we presented an idea for the Skolem-guided synthesis algorithm
that given a contract written in AADL would check realizability and derive an implementation,
but we did not implement it due to significant limitations in our Skolemizer~\cite{fedyukovich2015automated}.
\grigory{need to show other weaknesses of the previous work. Otherwise the reviewers may say ``not enough contributions''.}
%, can provide an implementation
%that is able to react to uncontrolled inputs provided by the system's
%environment, while satisfying the restrictions specified in the contract
%assumptions and guarantees.

The synthesis algorithm is an extension of our previous work on solving the problem of
realizability modulo infinite theories~\cite{Katis15:Realizability}, using a
model checking algorithm that has been formally verified in terms of its soundness for realizable
results~\cite{katis2015machine}.
\grigory{as we agreed earlier, there could be more description on k-induction}
 Given the inductive proof of realizability, we generate Skolem functions which are the
witnesses of strategies that a synthesized implementation can follow at each
step of execution.

\grigory{Let us invent a name for ``of our extension to \jkind's
realizability checking algorithm to support synthesis'' and use it everywhere.
The working name is \jkindsynt.}

We have implemented the synthesis
algorithm and exercised it in terms of its performance on several models. We
provide an informal proof of the algorithm's correctness regarding the
implementations that it produces, and discuss our experimental results.
\grigory{need more info here. some executive summary of the evaluation section.}

\fi 

%%% Local Variables:
%%% mode: latex
%%% TeX-master: "document"
%%% End:
