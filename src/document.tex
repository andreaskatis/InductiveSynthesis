\documentclass[orivec]{llncs}
\usepackage{amsmath}
\usepackage{listings}
\usepackage{amsfonts}
\usepackage{graphicx}
\usepackage{courier}
\usepackage{algorithmic}
\usepackage[table,xcdraw]{xcolor}
\usepackage{float}
\usepackage{hyperref}
\usepackage{mathtools}
\usepackage[framemethod=TikZ]{mdframed}
\usepackage[]{inputenc}
\usepackage[T1]{fontenc}
\usepackage{placeins}
\usepackage{amsmath,amssymb}
\usepackage{xspace}

%to solve the hyperref problem of jumping to wrong places
\usepackage[all]{hypcap}
\usepackage[framemethod=TikZ]{mdframed}
\usepackage{fancyvrb}
\usepackage{relsize}
\graphicspath{{images/}}
\usepackage[boxruled,shortend,linesnumbered,algo2e]{algorithm2e}  % algo2e = use \begin{algorithm2e}
\usepackage{float}

\newcommand{\aeval}{\textsc{AE-VAL}\xspace}

\renewcommand{\labelitemi}{\tiny$\blacksquare$}

\newcommand{\andreas}[1]{\textcolor{blue}{Andreas: #1}}
\newcommand{\mike}[1]{\textcolor{red}{Mike: #1}}
\newcommand{\andrew}[1]{\textcolor{green}{Andrew: #1}}
\newcommand{\grigory}[1]{\textcolor{brown}{Grigory: #1}}

\newenvironment{requirement}
{\vspace{0.05in}
 \begin{mdframed}[roundcorner=10pt,backgroundcolor=gray!20]}
{\end{mdframed}}

\begin{document}

\title{Synthesis from Assume-Guarantee Contracts using Skolemized Proofs of
Realizability}
\author{Andreas Katis\inst{1}, Grigory Fedyukovich\inst{2}, Andrew
Gacek\inst{3}, John Backes\inst{3}, Michael W. Whalen\inst{1}}%
\institute{
Department of Computer Science and Engineering,\\
 University of Minnesota, 200 Union Street, Minneapolis, MN 55455,USA\\
\email{katis001@umn.edu, whalen@cs.umn.edu}
\and
Computer Science and Engineering, University of Washington, Seattle, Washington,
USA\\
\email{grigory@cs.washington.edu}
\and 
Rockwell Collins Advanced Technology Center\\
400 Collins Rd. NE, Cedar Rapids, IA, 52498, USA\\
\email{\{andrew.gacek,john.backes\}@rockwellcollins.com}
}
\maketitle

\begin{abstract}
\grigory{I think, in the abstract, we need to be more enthusiastic about the contributions.
Currently, it reads like there is a little of novelty: we extend the prior work, the approach is similar to something else, the evaluation is simple and so on.
I believe, the abstract should sell the technique, and be precise about the positive evaluation points.
Also, I would avoid messages like ``In recent work'' and ``our realizability checking algorithm''
since the reviewers most likely are unaware of them.
Instead, I would describe a bit what is ``k- inductive proof'' and possibly mention that this proof 
is not necessarily to be given by your algorithm. This way, the synthesis algorithm 
will look more generic, and the paper's contributions will be stronger.
Finally, the abstract should not contain the summary of the paper (i.e., the last two sentences).}

Program synthesis is a particularly interesting area of current research in
artificial intelligence, and more recently in formal verification. The main idea
is generate efficient implementations for systems, using the system-specific
requirements as the only source of information. This is especially important for
the case of embedded systems that are meant to be used as leaf-level,
independent components of a bigger, more complex hierarchical architecture.

One of the very challenging problems in requirements engineering is for one to
decide whether the documented requirements are good enough to pivot the
development of an implementation that is guaranteed to meet them, given any circumstance.
This is mostly known as the implementability, or realizability problem.
Misconceptions and conflicts between requirements may occur during this process, and might not be
easy to detect without the use of sophisticated tools, meaning that the engineer
can end up with specification for which an implementation does not even exist.
This fact alone can impose a big overhead in the system's development cycle, both in time and in cost of development.
Having a proof of realizability  for our requirements though, among other
interesting requirement characteristics, directly implies that we can
construct an implementation for them. Furthermore, if used properly, the same
proof can be actually used to construct such an implementation automatically,
and thus solve the problem of program synthesis.

In the context of this paper, we propose a program synthesis algorithm for
requirements written in the form of an Assume-Guarantee contract, using the
Lustre specification language. The algorithm relies heavily on the proof of
the contract's realizability, using an approach that is very similar to k-induction model checking, with the additional use of quantifiers.
With the k-inductive proof as a guide, and a sophisticated tool for
extracting Skolem functions from $\forall\exists$-formulas, we can effectively synthesize
implementations that, by definition, are guaranteed to comply with the contract.
We have incorporated the main synthesis algorithm as an extension to the
realizability check provided in the JKind model checker, and furthermore
developed a compiler to translate these primitive implementations to the C
language. The resulting implementations in C are further being tested and
compared against the corresponding implementations provided by LustreC, a
compiler from Lustre to C implementations.

\end{abstract}


\section{Introduction}
Automated synthesis research is concerned with discovering efficient algorithms to construct candidate programs that are guaranteed to comply with predefined temporal specifications.  This problem has been well studied for propositional specifications, especially for (subsets of) LTL~\cite{gulwani2010dimensions}.  More recently, the problem of synthesizing programs for richer theories has been examined, including work in {\em template synthesis}~\cite{srivastava2013template}, which attempts to find programs that match a certain shape (the template), and {\em inductive synthesis}, which attempts to use counterexample-based refinement to solve synthesis problems~\cite{flener2001inductive}.  Such techniques have been widely used for stateless formulas over arithmetic domains~\cite{reynoldscounterexample}.
\textit{Functional synthesis} has also been effectively used to synthesize
subcomponents of already existing partial
implementations~\cite{kuncak2013functional}.

In this paper,
we propose a new approach that can synthesize programs for arbitrary {\em
assume/guarantee contracts} that do not have to conform to specific template
shapes or temporal restrictions. The contracts are described using
safety properties involving real arithmetic.  Although the
technique is not guaranteed to succeed or terminate, we have used it to successfully synthesize a range of programs over non-trivial contracts.
It is more general than previous approaches for temporal synthesis involving theories, supporting both arbitrary safety properties rather than ``stateless'' properties (unlike~\cite{reynoldscounterexample}) and arbitrary shapes for synthesized programs (unlike~\cite{srivastava2013template}).

Our approach is built on previous work determining the {\em realizability} of contracts involving infinite theories such as linear integer/real arithmetic and/or uninterpreted functions~\cite{Katis15:Realizability,katis2015machine}.  The algorithm, explained in Section~\ref{sec:synthesis}, uses a quantified variant of k-induction that can be checked by any SMT solver that supports quantification.  Notionally, it checks whether a sequence of states satisfy the contract of depth $k$ is sufficient to guarantee the existence of a successor state that satisfies the contract for an arbitrary input.  An outer loop of the algorithm increases $k$ until either a solution or counterexample is found.

The step from realizability to synthesis involves moving from the existence of a witness (as can be provided by an SMT solver such as Z3 or CVC4) to the witness itself.  For this, the most important obstacle is the (in)ability of the SMT solver to handle higher-order quantification. Fortunately, interesting directions to solving this problem have already surfaced, either by extending an SMT solver with native synthesis capabilities~\cite{reynoldscounterexample}, or by providing external algorithms that reduce the problem by efficient quantifier elimination methods~\cite{fedyukovichae}.  Our synthesis relies on our previous implementation for realizability checking and the skolemization procedure implemented in the \aeval tool~\cite{fedyukovichae}.


We combined the above ideas to create a reasonable sequential synthesis
approach, which we call \jkindsynt.  It applies the realizability checker
from~\cite{Katis15:Realizability} and then extracts a Skolem witness formula from the \aeval tool that can immediately be turned into a C program.  In order to support synthesis, several changes were required to the quantifier-elimination approach to produce Skolem {\em functions} rather than {\em relations}.  The main contributions of the work are therefore:
\begin{itemize}
	\item To the best of our knowledge, the first approach to synthesize implementations from Assume-Guarantee contracts containing temporal safety properties involving theories.
	\item A framework of extracting fine-grained Skolem functions for $\forall\exists$-formulas in Linear Real Arithmetic
	\item A prototype tool implementing the algorithm
	\item An experiment demonstrating the application of the tool on various benchmark examples
\end{itemize}

\noindent This paper presents the first full exposition of the idea, which was originally proposed in a workshop paper without an implementation or experiment~\cite{katis2016towards}.

In Section~\ref{sec:synthesis}, we provide the necessary background
definitions that are used in our synthesis algorithm, as well as an informal
proof of the algorithm's correctness. Section~\ref{sec:aeval} contains the
core formal notions on which the \aeval Skolemizer is based, as well
as the adjustments that were done for it to better support the needs of this
work. Section~\ref{sec:impl} provides detailed
source information for each one of the important
components of this work. Section~\ref{sec:experiment} presents our results on
using the algorithm to automatically generate leaf-level component implementations for different case studies.
In Section~\ref{sec:related} we give a brief historical background on the
related research work on synthesis, and we discuss potential future work in
Section~\ref{sec:futurework}. Finally, we conclude this paper in
Section~\ref{sec:conclusion}.

\iffalse



\grigory{The intro needs work. What's the main challenge? Why do we start with formal verification, and with synthesis or realizability checking?}
Formal verification is a well-established area of research, with ever increasing
popularity, in an attempt to provide better tools to software engineers during
the design and testing phases of a project and effectively reduce its overall
development cost. A particularly interesting problem in formal verification \grigory{this is not 100\% right, since program synthesis spreads far beyond program verification. There are many applications of synthesis that are not based on verification.} is
that of program synthesis aiming to construct efficient algorithms that can
automatically generate code which is guaranteed to behave correctly based on the
information provided by the user through formal or informal requirements.
\grigory{informal requirements? how come?}

Program synthesis has been applied in a significant amount of
diverse contexts, \grigory{citations?}, but to the best of our knowledge there is no application yet that ....%
 \grigory{what's the best novelty in our synthesis application?}
In this paper, we focus on the automated
generation of implementations for the leaf-level components of embedded
systems, using safety properties that are expressed in the form of an
Assume-Guarantee contract.
In recent work~\cite{katis2016towards}, we presented an idea for the Skolem-guided synthesis algorithm
that given a contract written in AADL would check realizability and derive an implementation,
but we did not implement it due to significant limitations in our Skolemizer~\cite{fedyukovich2015automated}.
\grigory{need to show other weaknesses of the previous work. Otherwise the reviewers may say ``not enough contributions''.}
%, can provide an implementation
%that is able to react to uncontrolled inputs provided by the system's
%environment, while satisfying the restrictions specified in the contract
%assumptions and guarantees.

The synthesis algorithm is an extension of our previous work on solving the problem of
realizability modulo infinite theories~\cite{Katis15:Realizability}, using a
model checking algorithm that has been formally verified in terms of its soundness for realizable
results~\cite{katis2015machine}.
\grigory{as we agreed earlier, there could be more description on k-induction}
 Given the inductive proof of realizability, we generate Skolem functions which are the
witnesses of strategies that a synthesized implementation can follow at each
step of execution.

\grigory{Let us invent a name for ``of our extension to \jkind's
realizability checking algorithm to support synthesis'' and use it everywhere.
The working name is \jkindsynt.}

We have implemented the synthesis
algorithm and exercised it in terms of its performance on several models. We
provide an informal proof of the algorithm's correctness regarding the
implementations that it produces, and discuss our experimental results.
\grigory{need more info here. some executive summary of the evaluation section.}

\fi

%%% Local Variables:
%%% mode: latex
%%% TeX-master: "document"
%%% End:

\section{Synthesis from Assume-Guarantee Contracts}
\label{sec:synthesis}

In this section we provide a summary of the formal background
that has already been established in previous work, regarding an algorithm that
is able to generate leaf-level component implementations using only the
information provided by the user through requirements expressed in the form of an
Assume-Guarantee contract. Our approach mainly supports the Linear Real
Arithmetic (LRA) theory, and to a certain extend the theory of integers (LIA),
mainly due to the limitations imposed by the underlying machinery. We
begin with a brief description of an Assume-Guarantee contract, and
move on to discuss the specifics of our program synthesis procedure,
which depends on our earlier work towards solving the problem of realizability
checking of contracts.
Finally, we enrich our formal definitions with an informal proof of the
algorithm's correctness in terms of the successfully synthesized
implementations.

\subsection{Assume-Guarantee Contracts}

In the context of requirements engineering, there have been a lot of proposed
ideas in terms of how requirements can be represented and expressed during
system design. One of the most popular ways to describe these requirements is through
the notion of an Assume-Guarantee contract, where the requirements are expressed
using safety properties that are split into two separate categories. The
\textit{assumptions} of the contract correspond to properties that restrict the
set of valid inputs a system can process, while the \textit{guarantees} dictate
what the system's behavior should be, using properties that precisely describe
the kinds of valid outputs that it may return to its environment.

\begin{figure}[H]
	\centering
	\includegraphics[width=\textwidth,height=\textheight,keepaspectratio]{real1-crop}    	
	\caption{Example of an Assume-Guarantee contract}
	\label{fg:example}
\end{figure}

As an illustrative example, consider the contract specified in
Figure~\ref{fg:example}. The component to be designed consists of two inputs,
$x$ and $y$ and one output $z$. If we restrict our example to the case of integer arithmetic,
we can see that the contract assumes that the inputs will never have the same value,
and requires that the component's output is a Boolean whose value depends on the comparison of the values of $x$ and $y$.
Also, notice that in the middle of the figure we depict the component using a
questionmark symbol. The questionmark is simply expressing the fact that during
the early stages of software development, the implementation is absent or exists only partially. This is particularly
important with respect to the problem of \textit{realizability}, where we try to
answer whether there exists an implementation that will satisfy the specific contract, under all circumstances. It is obvious that this is also
particularly important at the harder problem of \textit{program synthesis},
where the goal is to construct a witness of the contract's proof of realizability. In
Figure~\ref{fg:example}, one can easily answer that the contract is
\textit{realizable}, and therefore a synthesis procedure should be able to
provide us with an implementation. On the other hand, if we omit the contract's assumption, we can
safely say that the contract is \textit{unrealizable} as no implementation will
be able to provide a correct output in the case where $x=y$.

\subsection{Formal Preliminaries}
For the purposes of this paper, we are describing a system using the types
$state$ and $inputs$. Formally, an \textit{implementation}, i.e. a
\textit{transition system} can be described using a set of initial states $I(s)$ of type $state \rightarrow bool$, in addition to a transition relation $T(s,i,s')$ that
implements the contract and has the type $state \rightarrow inputs \rightarrow
state \rightarrow bool$.
 
An Assume-Guarantee contract can formally defined by two sets, a set of
\textit{assumptions} and a set \textit{guarantees}. The \textit{assumptions} $A$
impose constraints over the inputs, while the \textit{guarantees} $G$ used for
the corresponding constraints over the system's outputs and can be expressed as
two separate subsets $G_I$ and $G_T$, where $G_I$ defines the set of valid
initial states, and $G_T$ specifies the properties that need to be met during
each new transition between two states. Note that we do not necessarily expect
that a contract would be defined over all variables in the transition system,
but we do not make any distinction between internal state variables and outputs in the formalism.
This way, we can use state variables to, in some cases, simplify statements of guarantees.

\subsection{Realizability of Contracts}
The synthesis algorithm of this paper is essentially an extension on our
previous work on the realizabiility problem. Given the formal foundations above,
we expressed the problem of realizability using the notion of a state being
\textit{extendable}:

\begin{definition}[One-step extension]
\label{def:extend}
A state $s$ is extendable after $n$ steps, written $Extend_{n}(s)$, if
any valid path of length $n-1$ from $s$ can be extended in response to
any input. That is,
\begin{multline*}
\forall i_1, s_1, \ldots, i_n, s_n.\\ A(s, i_1) \land G_T(s, i_1, s_1)
\land \cdots \land
A(s_{n-1}, i_n) \land G_T(s_{n-1}, i_n, s_n)
\Rightarrow \\
\forall i.~ A(s_n, i) \Rightarrow \exists s'.~ G_T(s_n, i, s')
\end{multline*}
\end{definition}

The algorithm for realizability is using Definition~\ref{def:extend} in two
separate checks, that correspond to the two traditional cases exercised in
k-induction. For the \textit{BaseCheck}, we ensure that all initial states are
extendable in terms of any path of length $k<=n$, while the inductive step of
\textit{ExtendCheck} tries to prove that all valid states are extendable.
Therefore, we try to find the smallest $n$, for which the two following checks
hold:

\begin{equation}
\label{eq:sbcheck}
BaseCheck(n) = \forall k \leq n. (\forall s. G_I(s)
	  	\Rightarrow Extend_k(s))
\end{equation}

\begin{equation}
\label{eq:echeck}
ExtendCheck(n) = \forall s. Extend_n(s)
\end{equation}

The realizability checking algorithm has been used to effectively find cases
where the traditional consistency check failed to detect conflicts between
stated requirements in case studies of different complexity and importance. It
has also been formally verified using the Coq proof assistant in terms of its
soundness, for the cases where it reports that a contract is realizable.

\subsection{Program Synthesis from the proof of Realizability}

While the implemented algorithm on realizability provided us with meaningful
results during the verification of several contracts, the most apparent and
important outcome of this work was the fact that it could be effectively used as
the basis towards solving a more complex problem, which is that of
\textit{program synthesis}. Synthesis is defined as the process of automatically
deriving implementations, given a set of requirements specified by the user.
Since we are able to derive a proof regarding a contract's realizability,
i.e. a proof that an implementation exists for the specified
contract, we can use this proof in order to construct a witness
implementation that satisfies it. The limited power of SMT solvers
in terms of solving formulas containing nested quantifiers immediately ruled
out the prospect of using one as our primary synthesis tool. Fortunately, 
we are able to exploit our prior results in the scope of solving validity and 
Skolemizing $\forall\exists$-formulas (to be described in Sect.~\ref{sec:aeval}).

The idea behind our approach to solving the synthesis problem is
straightforward. Consider the checks~\ref{eq:sbcheck} and~\ref{eq:echeck} that
are used in the realizability checking algorithm. Both checks require
that the reachable states explored are extendable using
Definition~\ref{def:extend}.
The idea then is to decide if  $Extend_{n}(s)$ is valid and generate a witness 
for each of the $n$ times that we run \textit{BaseCheck} and a final witness 
for the inductive case in \textit{ExtendCheck}.

%GRIGORY: commented out the translation of A=> B => C into A /\ B => C since it is quite obvious

% Of course, 
%$Extend_{n}(s)$ as defined in~\eqref{def:extend} cannot
%be directly used for this purpose due to its form. This is not really an
%obstacle though, as we can rewrite the definition:
%
%\begin{multline*}
%\forall i_1, s_1, \ldots, i_n, s_n.\\ A(s, i_1) \land G_T(s, i_1, s_1)
%\land \cdots \land
%A(s_{n-1}, i_n) \land G_T(s_{n-1}, i_n, s_n)
%\Rightarrow \\
%\forall i.~ A(s_n, i) \Rightarrow \exists s'.~ G_T(s_n, i, s')
%\end{multline*}
%
%into an equivalent formula of the form $\forall \vec{x}.
%S(\vec{x}) \Rightarrow \exists \vec{y}. T(\vec{x},\vec{y})$ : 
%\begin{multline}
%	\label{ml:extendable2}
%		\forall i_1,s_1,\ldots,i_n,s_n,i. \\
%		A(s,i_1) \wedge G_T(s,i_1,s_1) \wedge \ldots \wedge
%		A(s_{n-1}, i_n) \wedge G_T(s_{n-1},i_n,s_n) \wedge A(s_n,i) \Rightarrow \\
%		\hspace{+2cm} \exists s'. G_T(s_n,i,s')
%	\end{multline}

\begin{figure}
\begin{small}
\begin{verbatim}
// for each variable in I or S,
//   create an array of size k.
//   then initialize initial state values
assign_GI_witness_to_S;
update_array_history;

// Perform bounded 'base check' synthesis
read_inputs;
base_check'_1_solution;
update_array_history;
...
read_inputs;
base_check'_k_solution;
update_array_history;

// Perform recurrence from 'extends' check
while(1) {
 read_inputs;
 extend_check_k_solution;
 update_array_history;
}
\end{verbatim}
\end{small}
\caption{Algorithm skeleton for synthesis}
\label{fig:algorithm}
\end{figure}

\noindent

Thus, we can construct the skeleton of an algorithm as shown in Figure~\ref{fig:algorithm}.  
We begin by creating an array for each input and history variable up to depth
$k$, where $k$ is the depth at which we found a solution to our realizability algorithm.
In each array, the zeroth element is the ``current'' value of the variable, the first element is the previous value, and the $(k-1)$'th value is the $(k-1)$-step previous value.
We then generate witnesses for each of the {\em BaseCheck} instances of
successive depth to describe the initial behavior of
the implementation up to depth $k$.  This process starts from the memory-free
description of the initial state ($G_I$).  There are two ``helper'' operations:
{\em update\_array\_history} shifts each array's elements one position forward
(the $(k-1)$'th value is simply forgotten), and {\em read\_inputs} reads the current values of inputs into the zeroth element of the input variable arrays.  Once the history is entirely initialized using the {\em BaseCheck} witness values, we enter a recurrence loop where we use the solution of the {\em ExtendCheck} to describe the next value of outputs.
 
 \andreas{Add proof of correctness here}



\newcommand{\such}{\,.\,}
\newcommand{\vx}{\vec{x}}
\newcommand{\vy}{\vec{y}}
\newcommand{\Land}{\bigwedge}
\newcommand{\Lor}{\bigvee}
\newcommand{\mbp}{\mathit{MBP}}
\newcommand{\unsat}{\textsc{unsat}}
\newcommand{\sat}{\textsc{sat}}
\newcommand{\valid}{\textsc{valid}\xspace}
\newcommand{\invalid}{\textsc{invalid}\xspace}
\newcommand*\rfrac[2]{{}^{#1}\!/_{#2}}
\newcommand{\tuple}[1]{\langle #1 \rangle}       % tuple (in mathmode)

\newcommand{\aevalalgorithm}{%
\begin{algorithm2e}[tb]
\SetAlgoSkip{}
\SetKwFor{While}{forever}{do}{}
%\SetAlgoNoLine
\SetKw{KwContinue}{continue}
\KwIn{$S(\vx), \exists \vy \such T(\vx,\vy)$.}
\KwOut{Return value $\in \{\valid, \invalid\}$ of ${S(\vx)\!\! \implies\!\! \exists \vy \such T(\vx,\vy)}$.}
\KwData{$\textsc{SmtSolver}$, counter $i$, models $\{m_i\}$, MBPs $\{T_{i}(\vx)\}$, conditions $\{\phi_i({\vx,\vy})\}$.}
\BlankLine
$\textsc{SmtAdd}(S(\vx))$; \\
$i \gets 0$; \\
\While{}{
$i$++; \\
%$res \leftarrow \textsc{SmtSolve}()$\label{alg:check_unsat_s}; \\
\lIf(\label{alg:returnUnsat}){$(\isUnSat(\textsc{SmtSolve}()))$}{\Return \valid }
$\textsc{SmtPush}()$; \\
$\textsc{SmtAdd}(T(\vx,\vy))$; \\
%$res \gets \textsc{SmtSolve}()$\label{alg:find_matching_ass};\\
\lIf(\label{alg:returnSat}){$(\isUnSat(\textsc{SmtSolve}()))$}{\Return \invalid }
$m_i \gets \textsc{SmtGetModel}()$\label{alg:model};\\ 
%$E \leftarrow extrapolate(m)$;\\
$(T_{i},\phi_i({\vx,\vy}))\! \gets\! \textsc{GetMBP}(\vy, m_i, T(\vx,\vy)))$\label{alg:proj};\\
$\textsc{SmtPop}()$;\\
$\textsc{SmtAdd}(\neg {T_{i}})$\label{alg:block}; \\
}
%\Return $res$;
\caption{\aeval \Big($S(\vx), \exists \vy \such T(\vx,\vy)$\Big), cf.~\cite{fedyukovich2015automated} }
\label{alg:ae_val}
\end{algorithm2e}
}

\newcommand{\localfactoralg}{%
\begin{algorithm2e}[t!]
\SetAlgoSkip{}
\SetInd{0.4em}{0.4em}
\SetKwFor{ForAll}{forall}{do}{}
\SetKwFor{For}{for}{do}{}
\KwIn{$y_j \in \vy$, local Skolem relation $\phi(\vx,\vy) = \Land_{y_j \in \vy}(\psi_{y_j}(\vx,y_{j},\ldots, y_{n}))$, Skolem functions $y_{j+1} = f_{j+1}(\vx),\ldots, y_{n} = f_{n}(\vx)$.}
\KwData{Factored formula $\pi_{y_j}(\vx,y_{j}) = L_{y_j} \land U_{y_j} \land M_{y_j} \land V_{y_j} \land E_{y_j} \land N_{y_j}$; assuming (for simplicity of presentation) that $M_{y_j} = \varnothing \land  V_{y_j} = \varnothing$.}
\KwOut{Local Skolem function $y_j = f_j(\vx)$.}
\BlankLine
\For{$(i = n; i > j; i++)$}{
  $\psi_{y_j}(\vx,y_{j},\ldots,y_{n}) \gets \textsc{Substitute}(\psi_{y_j}(\vx,y_{j},\ldots,y_{n}) , y_i, f_i(\vx))$\label{alg:loc_subst};\\
}
\BlankLine

$\pi_{y_j}(\vx,y_{j}) \gets \psi_{y_j}(\vx,y_{j},\ldots,y_{n})$\label{alg:elim_compl};\\
\BlankLine

\lIf(){$(E_{y_j} \neq \varnothing)$}{\Return $E_{y_j}$}

$\pi_{y_j}(\vx,y_{j}) \gets \textsc{Merge}(L_{y_j}, \mathit{MAX}, \pi_{y_j}(\vx,y_{j}))$;\\
$\pi_{y_j}(\vx,y_{j}) \gets \textsc{Merge}(U_{y_j} , \mathit{MIN}, \pi_{y_j}(\vx,y_{j}))$;\\
\BlankLine

\lIf(){$(U_{y_j} = \varnothing \land N_{y_j} = \varnothing)$}{\Return $\textsc{Rewrite}(L_{y_j} , \mathit{GT}, \pi_{y_j}(\vx,y_{j}))$}

\lIf(){$(L_{y_j} = \varnothing  \land N_{y_j} = \varnothing)$}{\Return $\textsc{Rewrite}(U_{y_j} , \mathit{LT}, \pi_{y_j}(\vx,y_{j}))$}
\BlankLine

\lIf(){$(N_{y_j} = \varnothing)$}{\Return $\textsc{Rewrite}(L_{y_j} \land U_{y_j} , \mathit{MID}, \pi_{y_j}(\vx,y_{j}))$}

\Return $\textsc{Rewrite}(L_{y_j} \land U_{y_j} \land N_{y_j}, \mathit{FMID}, \pi_{y_j}(\vx,y_{j}))$\label{alg:loc_inst};\\
\caption{$\textsc{ExtractSkolemFunction}(y_j, \phi(\vx,\vy)$)}
\label{alg:loc}
\end{algorithm2e}
}

\section{Witnessing existential quantifiers with \aeval}
\label{sec:aeval}

Quantifier elimination is a decision procedure that turns a quantified formula into an equivalent quantifier-free formula.
In addition, the quantifier elimination algorithms are often able to discover a Skolem function that represents witnesses for the existentially quantified individual variables (e.g.,~\cite{DBLP:conf/cav/BalabanovJ11,DBLP:journals/sttt/KuncakMPS13,KLXJOIA,Chakraborty15}).
%
Various tasks in verification and synthesis~\cite{DBLP:conf/fmcad/CimattiGMT13,DBLP:conf/popl/BeyeneCPR14,DBLP:conf/nfm/GasconT14} rely on efficient techniques to remove existential quantifiers from formulas in first order logic, thus adjusting the task to be decided by an SMT solver.
In particular, \emph{functional synthesis} aims at computing a function that meets a given input/output relation.
A function with an input $x$ and an output $y$, specified by a relation $f(x,y)$, can be constructed as a by-product of deciding validity of the formula $\forall x \exists y \such f(x,y)$.
Due to a well-known \emph{AE-paradigm} (also referred to as \emph{Skolem paradigm}~\cite{DBLP:conf/popl/PnueliR89}),
the formula $\forall x \exists y \such f(x,y)$ is equivalent to the formula $\exists \mathit{sk} \; \forall x \such f(x, \mathit{sk}(x))$, which means existence of a Skolem function $\mathit{sk}$, such that $f(x,\mathit{sk}(x))$ holds for every $x$.
Thus the key feature in modern quantifier elimination approaches is their ability to produce witnessing Skolem function.

In the rest of the section, we briefly describe the prior work on \aeval to be able (in Sect.~\ref{sec:new}) to present the key contributions on delivering Skolem functions appropriate for the program synthesis from proofs of realizability.

\subsection{Model-Based Projection for Linear Rational Arithmetic}
\label{sec:mbp}

Quantifier elimination of a formula $\exists \vy \such T(\vx,\vy)$ is an expensive procedure that typically proceeds by enumerating all models of an extended formula $T(\vx,\vy)$.
However, in some applications, the quantifier-free formula, fully equivalent to $\exists \vy \such T(\vx,\vy)$, is not even needed.
Instead, it is enough to operate by (possibly incomplete) sets of models.
This idea relies on some notion of projection that under-approximates existential quantification.
In this section, we consider a concept of Model-Based Projections (MBP), recently proposed by~\cite{komuravelli2014smt,Dutertre}.

%In the following, we use vector notation to denote sets of variables (and set-theoretic operators of \emph{subset} $\vu \subseteq \vx$, \emph{complement} $\vx_{\vu} = \vx \setminus \vu$, \emph{union} $\vx = \vu \cup \vx_{\vu}$).
\begin{definition}
\label{def:mbp}
An $\mathit{MBP}_{\vy}$ is a function from models of
$T(\vx,\vy)$ to $\vy$-free formulas
iff:
\begin{gather}
\text{if }m\models T(\vx,\vy) \text{ then } m\models \mathit{MBP}_{\vy}(m,T) \label{mbp.cond1}\\
\mathit{MBP}_{\vy}(m,T) \!\implies\! \exists \vy \such T(\vx,\vy) \label{mbp.cond2}
\end{gather}
\end{definition}

There are finitely many MBPs for fixed  $\vy$ and $T$ and different models $m_1,\ldots,m_n$ (for some $n$):
$T_{1}(\vx),  \ldots, T_{n}(\vx)$, such that
$\exists \vy \such T(\vx,\vy) = \Lor_{i=1}^{n} T_{i}(\vx)$. 

A possible way of implementing an MBP-algorithm was proposed in~\cite{komuravelli2014smt}.
It is based on Loos-Weispfenning (LW)  quantifier-elimination method~\cite{loos1993applying} for Linear Rational Arithmetic (LRA).
Consider formula $\exists \vy \such T(\vx,\vy)$, where $T$ is quantifier-free.
In our simplified presentation,
$\vy$ is singleton, $T$ is in Negation Normal Form (that allows the operator $\neg$ to be applied only to variables), and $y$ appears in the literals only of the form ${y=e}$, ${l<y}$ or ${y<u}$, where $l,u,e$ are $y$-free.
LW states that the equation~\eqref{eq:formula_rewrite} holds:

\begin{equation}
  \exists y \such T (\vx) \equiv \Big( \Lor_{(y = e) \in \mathit{lits}(T)}{T[e]} \lor
  	\Lor_{(l < y) \in \mathit{lits} (T)}{T [l + \epsilon]} \lor
	T[-\infty] \Big)\label{eq:formula_rewrite}
\end{equation} 
\smallskip  

In~\eqref{eq:formula_rewrite}, $\mathit{lits}(T)$ denote the set of literals of $T$, $T[\cdot]$ stands for a \emph{virtual substitution} for the literals containing $y$.
In particular, $T[e]$ substitutes exact values of $y$ ($y=e$), $T[l+\epsilon]$ substitutes the intervals ($l < y$) of possible values of $y$, $T[-\infty]$ substitutes the rest of the literals.
Consequently, a function $\mathit{LRAProj_{T}}$ is an implementation of the $\mathit{MBP}$ function for~\eqref{eq:formula_rewrite}:%
%
\begin{equation}
\begin{aligned}
\label{case:proj_define}
&\mathit{LRAProj}_{T}(m) = \left\{
\begin{array}{ll}
T[e], 			& \mbox{if}\ (y=e) \in \mathit{lits}(T) \land 
			m \models (y = e)\\
T[l+ \epsilon],	& \mbox{else if}\ (l < y) \in \mathit{lits}(T) 
			\land m \models (l < y) \land \\
			& \forall (l'\!<\!y)\!\in\!\mathit{lits}(T) \such \!m\!\models\!\big((l'\!<\!y)\!\!\implies\!\!(l'\! \le\! l)\big)\\
T[-\infty], 		& \mbox{otherwise}	
\end{array}
\right.
\end{aligned}
\end{equation}





\subsection{Validity and Skolem extraction}
\label{sim:check}

\aevalalgorithm  

Skolemization (i.e., introducing Skolem functions) is a well-known
technique for removing existential quantifiers in first order formulas.
%More specifically:
%
%\begin{definition}[cf.~\cite{skolem}]
%\label{def:sk_fun}
Given a formula ${ \exists y \such \psi(\vx, y)}$,
a~\emph{Skolem function} for $y$, $\mathit{sk}_{y}(\vx)$ is a function such that
$\exists y \such \psi(\vx,y)\!\iff\!\psi (\vx, \mathit{sk}_{y} (\vx))$.
%\end{definition}
We generalize the definition of a Skolem function for the case of a
vector of existentially quantified variables $\vy$, by relaxing the
relationships between elements of $\vx$ and $\vy$.
Given a formula ${\exists \vy \such \Psi(\vx, \vy)}$, a~\emph{Skolem relation} for $\vy$ is a relation ${\mathit{Sk}_{\vy} (\vx, \vy)}$ such that 1) $\mathit{Sk}_{\vy} (\vx, \vy) \implies \Psi (\vx, \vy)$ and 2) $\exists \vy \such \Psi(\vx, \vy)\!\iff\!\mathit{Sk}_{\vy} (\vx, \vy)$.
  

The algorithm \aeval for deciding validity and Skolem extraction assumes that a formula $\Psi$ can be transformed
into the form ${\exists \vy \such \Psi(\vx, \vy)} \equiv {S(\vx)
  \!\implies\! \exists \vy \such T(\vx,\vy)}$, where $S(\vx)$ has only
existential quantifiers, and $T(\vx, \vy)$ is quantifier-free.
%
\aeval partitions
the formula, and searches for a witnessing local Skolem relation of
each partition.  \aeval iteratively constructs a set of MBPs $\{T_i(\vx)\}$, each of which 
is connected with a so called local Skolem relation $\phi_i(\vx,\vy)$, such that
$\phi_i(\vx,\vy) \!\implies\! (T_{i}(\vx) \!\iff\!
  T(\vx,\vy))$ (i.e., that make the corresponding projections equisatisfiable with $T$).
While the pseudocode of \aeval is shown in Alg.~\ref{alg:ae_val}, we refer the reader to~\cite{fedyukovich2015automated} for
more detail.

%\aeval is shown in Alg.~\ref{alg:ae_val}.  Given formulas
%$S(\vx)$ and ${\exists \vy \such T(\vx, \vy)}$, it
%decides validity of ${S(\vx)\! \implies\! \exists \vy \!\such\!
%  T(\vx, \vy)}$.  \aeval enumerates the
%models of $S \land T$ and blocks them from $S$.
%In each iteration $i$, it first checks whether
%$S$ is non-empty (line~3) and then looks for a model $m_i$ of $S
%\land T$ (line~\ref{alg:model}).  If $m_i$ is found, \aeval
%gets a projection $T_i$ of $T$ based on $m_i$ (line~\ref{alg:proj})
%and blocks all models contained in $T_i$ from $S$
%(line~\ref{alg:block}).  The algorithm iterates until either it
%finds a model of $S$ that can not be extended to a
%model of $T$ (line~\ref{alg:returnSat}), or all models of $S$
%are blocked (line~\ref{alg:returnUnsat}). In the first case, the input
%formula is invalid. In the second case, every model of $S$ has been extended
%to some model of $T$, and the  formula is valid. 

\newcommand{\skolemcases}{%
\begin{equation}
\label{case:skolem}
\mathit{Sk}_{\vy} (\vx, \vy) \equiv
\begin{cases}
  \phi_{1} (\vx, \vy)  & \text{if } T_1 (\vx) \\
    \phi_{2} (\vx, \vy)  & \text{else if } T_2 (\vx)\\
  \cdots &\text{\qquad else }\cdots \\
  \phi_{n} (\vx, \vy) & \text{\qquad\qquad else } T_n (\vx) \\
\end{cases}
\end{equation}
}

A Skolem relation
${\mathit{Sk}_{\vy} (\vx, \vy)}$ by \aeval maps each
model of $S(\vx)$ to a corresponding model of $T(\vx,\vy)$.
Intuitively, $\phi_i$ maps each model of $S \land T_{i}$ to a model of $T$.
Thus, in order to define the  Skolem relation $\mathit{Sk}_{\vy}(\vx, \vy)$ it is enough to 
match each $\phi_i$ against the corresponding $T_{i}$:

\skolemcases


\subsection{Refining Skolem Relations into Skolem Functions}
\label{sec:new}

\localfactoralg


Since \aeval is an extension of the MBP-algorithm mentioned in Sect.~\ref{sec:mbp},
each $\phi_i$ (in~\eqref{case:skolem}) is constructed from the substitutions made in $T$ to produce $T_{i}$.
Furthermore, 
%(is a condition under which $T$ is equisatisfiable with $T_{i}$).
%We assume that each $\phi_i$ is in the Cartesian form, i.e., a conjunction of
%terms, in which each $y_j \in \vy$ appears at most once:
%\begin{equation}
%\label{eq:phi_conj}
%\phi_i(\vx,\vy) = \Land_{y_j \in \vy}(\psi_{y_j}(\vx,y_{j},\ldots,y_{n}))
%\end{equation}
%
%By construction, each local Skolem relation $\phi(\vx,\vy)$ has a form $\Land_{y_j \in \vy}(\psi_{y_j}(\vx,y_{j},\ldots,y_{n}))$.
%Since
each MBP in \aeval is constructed iteratively for each variable $y_j \in \vy$.
Thus, $y_j$ may depend on the variables of $y_{j+1},\ldots, y_{n}$ that are still not eliminated in the current iteration $j$.
%Each $\psi_{y_j}(\vx,y_{j},\ldots,y_{n})$ is the conjunction $\psi_{y_j}(\vx,y_{j},\ldots,y_{n}) = \Land_{i}(cl_i(\vx,y_{j},\ldots,y_{n}))$, where each $cl_i$ is  an (in)equality.

Inequalities in a Skolem relation are the enemies \arie{enemies?!} of program synthesis.
Indeed, the final implementation should contain assignments to each
existentially quantified variable, which are in general difficult to
get. \arie{What is meant by ``difficult''. Difficult computationally,
  difficult for the current algorithm, etc?}
The Skolem relation provided by \aeval should be post-processed to get rid of inequalities.
We formalize this procedure as finding a Skolem function $f_{y_j}(\vx)$ for each $y_j\in \vy$, such that $(y_j = f_{y_j}(\vx)) \!\implies\! \exists y_{j+1},\ldots,y_{n} \such \psi_{y_j}(\vx,y_{j},\ldots,y_{n}) $.
The key idea is presented in Alg.~\ref{alg:loc}.
The algorithm is applied separately for each $y_j \in \vy$, starting from $y_n$ to $y_1$.
For each $y_j$, assume, we already established Skolem functions $f_{j+1}(\vx),\ldots,f_{n}(\vx)$ for variables $y_{j+1},\ldots,y_n$ in the previous runs of the algorithm. 
%
First, the algorithm substitutes each appearance of variables $y_{j+1},\ldots, y_{n}$ in $\psi_{y_j}$ by $f_{n}(\vx),\ldots,f_{j+1}(\vx)$.
%If for some variable there is no Skolem function to substitute, the algorithm halts with nothing (line~\ref{alg:loc_ret_none}).
Second, %the algorithm uses standard LRA-rules to normalize $\pi_{y_j}(\vx,y_{j})$ into the form $\Land_k \big(y_{j} \sim f_k(\vx)\big)$, i.e., conjunction of
%expressions, left-hand-sides of which are reserved for $y_j$ and ${\sim
%\in \{<, \le, =, \ge, >\}}$.
%For this, it uses the method \textsc{Rewrite} (line~\ref{alg:loc_rewrite}) that rewrites each clause using the following rule (where $g,h$ - are functions over $\vx$, $p,q$ - rational numbers, $\mathit{sgn}$ - a function, returning the sign of the rational number):
%
%\begin{equation}
%{\Big( (g(\vx)+ p\!\times\!y_j}) \sim ( {h(\vx) + q\!\times\!y_j})\Big) \!\implies\! \Big(\big(\mathit{sgn}(p-q)\times y_j \big) \sim \big( -\frac{g(\vx)}{|p-q|} + \frac{h(\vx)}{|p-q|} \big)\Big)  \notag
%\end{equation}
%Finally 
the algorithm gets rid of inequalities by transforming them into
equalities, thus producing a Skolem function. \arie{The first part
  (line~2) is clear, the second part is not. Elaborate more which part
  of the algorithm are responsible for which inequality
  elimination. The sets $L$, $U$, $M$, $V$, $E$, $N$, are not
  defined. It is not clear why $M$ and $V$ are present but assumed to
  be empty.}
In the rest of the section, we show several fundamental rules and properties behind this operation.
For simplicity, we omit some straightforward details on dealing with
non-strict inequalities ($y_j \ge l(\vx)$ and $y_j \le u(\vx)$) since
they are similar strict inequalities ($y_j > l(\vx)$ and $y_j < u(\vx)$).

%If there are exist some, $f(\vx)$ defines some interval for the values for $y_j$.
%Method \textsc{SkolemFunction} obtains equalities 
%rewrites each clause using the following rules:

\grigory{the rest can be compressed if needed}.

\begin{lemma}
After all substitutions at line~\ref{alg:elim_compl} of Alg.~\ref{alg:loc}, each $\psi_{y_j}(\vx,y_{j},\ldots,y_{n})$ is a conjunction of the form
$L_{y_j} \land U_{y_j} \land M_{y_j} \land V_{y_j} \land E_{y_j} \land N_{y_j}$ where
%
$L_{y_j} = \Land_{l}(y_j > l(\vx))$, 
$U_{y_j} = \Land_{u}(y_j < u(\vx))$, 
$M_{y_j} = \Land_{l}(y_j \ge l(\vx))$, 
$V_{y_j} = \Land_{u}(y_j \le u(\vx))$, 
$E_{y_j} = \Land_{e}(y_j = e(\vx))$, 
$N_{y_j} = \Land_{n}(y_j \neq n(\vx))$.
%
\end{lemma}
\begin{proof}
Follows directly from~\eqref{case:proj_define}.
\qed
\end{proof}

The procedure to extract a Skolem function out of a Skolem relation proceeds by analyzing terms in $L_{y_j}$, $U_{y_j}$, $M_{y_j}$, $V_{y_j}$,$E_{y_j}$ and $N_{y_j}$.
If there is at least one conjunct $(y_j = e(\vx)) \in E_{y_j}$ then $(y_j = e(\vx))$ itself is a Skolem function.
Otherwise, the algorithm creates it from the following primitives.

\begin{definition}
Let $l(\vx)$ and $u(\vx)$ be two linear terms,
then operators $\mathit{MAX}$, $\mathit{MIN}$, $\mathit{MID}$, $\mathit{LT}$, $\mathit{GT}$ are defined as follows:
\begin{align*}
\mathit{MAX}(l, u) (\vx) &= ite (l (\vx) < u (\vx), u(\vx), l(\vx)) \notag &
\mathit{MIN}(l, u) (\vx) &= ite (l (\vx) < u (\vx), l(\vx), u(\vx)) \notag \\
\mathit{LT} (u) (\vx) &= u (\vx) -1 \notag &
\mathit{GT} (l) (\vx) &= l (\vx) + 1 \notag \\
\mathit{MID}(l, u) (\vx) &= \frac{l(\vx) + u(\vx)}{ 2} \notag 
\end{align*}
\end{definition}

\begin{lemma}
If $L_{y_j}$ consists of $n>1$ conjuncts then it is equivalent to $y_j > \mathit{MAX} (l_1, \mathit{MAX} (l_2,\ldots \mathit{MAX} (l_{n-1}, l_n) )) (\vx)$.
If $U_{y_j}$ consists of $n>1$ conjuncts then it is equivalent to $y_j < \mathit{MIN} (u_1, \mathit{MIN} (u_2,\ldots \mathit{MIN} (u_{n-1}, u_n) ))(\vx)$.
%If  $M_{y_j}$ consists of $n>1$ conjuncts then it is equivalent to $y_j \ge \mathit{MAX} (l_1, \mathit{MAX} (l_2,\ldots \mathit{MAX} (l_{n-1}$, $ l_n) )) (\vx)$.
%If  $V_{y_j}$ consists of $n>1$ conjuncts then it is equivalent to $y_j \le \mathit{MIN} (u_1$, $\mathit{MIN} (u_2,\ldots \mathit{MIN} (u_{n-1}, u_n) )) (\vx)$.
\end{lemma}
Similar for $M_{y_j}$ and for $V_{y_j}$. From this point on, with out
loss of generality, we assume that each $L_{y_j}$, $U_{y_j}$,
$M_{y_j}$, $V_{y_j}$ have at most one conjunct.\arie{Check}

\begin{lemma}
If $L_{y_j}$ consists of 1 conjunct and the rest of $U_{y_j}$, $M_{y_j} $,$V_{y_j}$,$E_{y_j}$ and $N_{y_j}$ are empty then the Skolem can be rewritten into $y_j = \mathit{GT} (l)(\vx)$.
\end{lemma}
Similar for $U_{y_j}$ (Skolem  rewritten into $y_j = \mathit{MID} (u)(\vx)$).

\begin{lemma}
If $L_{y_j}$, $U_{y_j}$  consist of 1 conjunct each, and the rest of $M_{y_j} $,$V_{y_j}$, $E_{y_j}$ and $N_{y_j}$ are empty then the Skolem can be rewritten into $y_j = \mathit{MID} (l, u)(\vx)$. 
\end{lemma}
Similar for $M_{y_j}$ and $V_{y_j}$, and for combinations with $L_{y_j}$ and $U_{y_j}$.


\begin{lemma}
  \arie{It is not clear where $h$ comes from}
If $L_{y_j}$, $U_{y_j}$ and $N_{y_j}$ consist of 1 conjunct each and the rest of $M_{y_j} $,$V_{y_j}$,$E_{y_j}$ and $N_{y_j}$ are empty then the Skolem can be rewritten into $y_j = \mathit{FMID} (l, u, h)(\vx)$, where%
\begin{align}
\mathit{FMID}(l, u, h) (\vx) = ite (&\mathit{MID}(l, u) (\vx) = h(\vx),  \notag \\
                          &\mathit{MID}(l, \mathit{MID}(l, u)) (\vx), \notag \\
                          &\mathit{MID}(l, u) (\vx)) \notag 
\end{align}
\end{lemma}
Similar for $M_{y_j}$ and $V_{y_j}$. 
For bigger number of conjuncts of $N_{y_j}$, the Skolem gets rewritten
in a similar way cascadically\arie{Not a word}.


\begin{lemma}
If $L_{y_j}$, $N_{y_j}$  consist of 1 conjunct each, and the rest of $M_{y_j} $,$V_{y_j}$, $E_{y_j}$ and $U_{y_j}$ are empty then the Skolem can be rewritten into $y_j = \mathit{FMID} (l, GT(l))(\vx)$. 
\end{lemma}
Similar for $M_{y_j}$.


\begin{lemma}
If $U_{y_j}$, $N_{y_j}$  consist of 1 conjunct each, and the rest of $M_{y_j} $,$V_{y_j}$, $E_{y_j}$ and $L_{y_j}$ are empty then the Skolem can be rewritten into $y_j = \mathit{FMID} (LT(u), u)(\vx)$. 
\end{lemma}
Similar for $V_{y_j}$.


\begin{theorem}[Soundness]
Iterative application of Alg.~\ref{alg:loc} to all variables $y_n,\ldots,y_1$ returns a local Skolem function to be used in~\eqref{case:skolem}.
\end{theorem}
\begin{proof}
Follows from the case analysis that applies the lemmas above.
\qed
\end{proof}

%%% Local Variables:
%%% mode: latex
%%% TeX-master: "document"
%%% End:

\section{Case Studies}

\subsection{A simple controller example}

In this section, we provide an illustrative example of how the synthesis
algorithm creates a simple implementation from specifications describing the
constraints that a controller must meet. The specification is written in the
Lustre language, and can be seen in Figure~\ref{fig:example}. The controller
is used to maintain an appropriate level of \textit{speed} at each next step of
its execution, using two auxillary signals, called \textit{plus} and
\textit{minus} to help determine future decisions on acceleration or deceleration. There is
only one input to this example, called \textit{diff}, and is used to compute
the amount by which the \textit{speed} value changes with each new state.

The specification is composed of an auxillary node called Sofar, that is a
custom operation on a boolean variable to capture whether it has been
historically true up to and including the current step. The rest of the nodes
defined essentially cover the assumptions and the guarantees that the contract
contains. The node \textit{Environment} is used to describe restrictions on the
input variable \textit{diff}, while the system's correct response is captured by the
node \textit{Property}. The \textit{top} node of the specification is used as
the main block of this program, and combines the two constraints to effectively
define the structure of the final property that the system should be respecting
at all states during its execution.

\andreas{add discription of Controller node. Add text for synthesis case}

 \begin{figure}[H]
 \begin{lstlisting}[basicstyle=\ttfamily\footnotesize]
--
-- Source Bertrand Jeannet, NBAC tutorial
--

node Sofar( X : bool ) returns ( Sofar : bool );
let
    Sofar = X -> X and pre Sofar;
tel

node Environment(diff: int; plus,minus: bool) returns (ok: bool);
let
  ok = (-4 <= diff and diff <= 4) and 
     (if (true -> pre plus) then diff >= 1 else true) and
     (if (false -> pre minus) then diff <= -1 else true);
tel

node Controller(diff: int) returns (speed: int; plus,minus: bool);
let
  speed = 0 -> pre(speed)+diff;
  plus = speed <= 9;
  minus = speed >= 11;
tel

node Property(speed: int) returns (ok: bool);
var cpt: int;
    acceptable: bool;
let
  acceptable = 8 <= speed and speed <= 12;
  cpt = 0 -> if acceptable then 0 else pre(cpt)+1;
  ok = true -> (pre cpt<=7);
tel

--@ ensures OK;
node top(diff:int) returns (OK: bool);
var speed: int; 
    plus,minus,realistic: bool;
let
  (speed,plus,minus) = Controller(diff);
  realistic =  Environment(diff,plus,minus);

  OK = Sofar( realistic and 0 <= speed and speed < 16 ) => Property(speed);
  --%PROPERTY OK;
  --%MAIN;
tel 
\end{lstlisting}
 \caption{Specification for a Controller in Lustre}
 \label{fig:example}
 \end{figure}
 

\label{sec:cstudies}


\section{Related Work}

Research in the field of program synthesis attributes its origins in the 1970s,
when Zohar Manna and Richard Waldinger first introduced a synthesis procedure
using theorem proving.~\cite{manna1971toward}. Almost two decades later, Amir
Pnueli and Roni Rosner were the first to propose a way to synthesize
implementations for temporal specifications~\cite{Pnueli89}. This work also
involved the first formal definition of a reactive system's realizability,
defined by the authors using the term implementability.

Since then, a vast variety of techniques have been developed. Efficient
algorithms were proposed for subsets of propositional LTL
\cite{Klein10,tomita2016safraless,ehlers2010symbolic,cheng2016structural} simple
LTL formulas \cite{Bohy12,hagihara2016simple,Tini03}, as well as other temporal
logics \cite{benevs2012factorization,monmege2016real,Hamza10}, such as SIS \cite{Aziz95}.
Component-based approaches have also been explored in~\cite{Chatterjee07,dammyou}.

Sumit Gulwani in 20120 published a survey on which he discribed the
potential future directions of program synthesis research~\cite{gulwani2010dimensions}.
The approaches that have been proposed are many, and differ on many aspects,
either in terms of the specifications that are being exercised, or the reasoning
behind the synthesis algorithm itself. Template-based
synthesis~\cite{srivastava2013template} is focused on the exploration of
programs that satisfy a specification that is refined after each
iteration, following the basic principles of deductive synthesis. Inductive
synthesis is an active area of research where the main goal
is the generation of an inductive invariant that can be used to describe the
space of programs that are guaranteed to satisfy the given
specification~\cite{flener2001inductive}.
This idea is mainly supported by the use of SMT solvers to guide the invariant
refinement through traces that violate the requirements, known as
counterexamples. Recently published work on extending SMT solvers with
counterexample-guided synthesis shows that they can eventually be
used as an alternative to solving the problem under certain domains of
arithmetic~\cite{reynoldscounterexample}. Reactive synthesis has also
been explored in the context involving propositional formulas for safety
specifications~\cite{bloem2016satisfiability}. Finally, functional synthesis is
used in applications where only a partial implementation exists, and the user needs an automated way to complete the missing parts of the program~\cite{kneuss2013integrating}.

A rather important contribution in the area is the recently published work by
Leonid Ryzhyk and Adam Walker~\cite{ryzhykdeveloping}, where they share their
experience in developing and using a reactive synthesis tool for controllers in
an industrial environment. While the authors emphasize that the research on
program synthesis is still at a very early stage for the technique to be
essential to industrial applications, they note its potential advantages in terms
of improving the overall development cycle of software.

To the best of our knowledge our work is the first complete attempt on providing
a synthesis algorithm for an assume-guarantee framework, using infinite theories.
We take advantage of a sophisticated solver that is able to reason about the
validity of the intermediate formulas that construct a k-inductive proof, as
well as provide witnesses for these formulas through the use of Skolem
functions. The ability to express contracts that support ideas from many
categories of specifications, such as template-based and temporal properties,
increases the potential applicability of this work to multiple subareas on
synthesis research.
\label{sec:related}

\section{Future Work}
\label{sec:futurework}

The meaningful results of our work so far on the synthesis from Assume-Guarantee
contracts have also provided a solid ground towards extending and improving the
involved algorithms in the future. A particularly important milestone is to
eventually switch to a more efficient algorithm, where we endorse the core
idea of Property Directed Reachability~\cite{een2011efficient,bradley11},
using efficient ways to further enhance the algorithm's performance through the
use of implicit abstractions~\cite{cimatti2014ic3} to further reduce the search
space of the algorithm. This will also help our original work on realizability
checking, by improving the unsoundness of our unrealizable results. Another
promising idea here is the use of Inductive Validity Cores (IVCs)~\cite{Ghass16}, whose main purpose is to effectively pinpoint the absolutely necessary model elements in a generated proof. We can potentially use
the information provided by IVCs as a preprocessing tool to reduce the size of
the original specification, and hopefully the complexity of the realizability
proof. Of course, a few optimizations can be further implemented in terms of
\aeval's specific support on proofs of realizability and finally, a very
important subject is the further improvement of the compiler that we created to
translate the Skolem functions into C implementations, by introducing
optimizations like common subexpression elimination.

Another important goal is that of supporting additional theories, and primarily
LIA, which is currently not fully supported by \aeval's model based projection
technique. Finally, another potential optimization that could effectively reduce 
the algorithm's complexity is the further simplification of the transition
relation that we are currently using, by reducing its complicated form through the
mapping of common subexpressions on different conditional
branches. This will also have a direct impact on the skolem relations retrieved
by \aeval, reducing their individual size and improving, thus, the final
implementation in terms of readability as well as its usability as an
intermediate representation to the preferred target language.


\section*{Acknowledgments}
This work was funded by DARPA and AFRL under contract 4504789784 (Secure Mathematically-Assured Composition of Control Models), and by NASA under contract NNA13AA21C (Compositional Verification of Flight Critical Systems), and by NSF under grant CNS-1035715 (Assuring the safety, security, and reliability of medical device cyber physical systems).
\bibliography{document}
\bibliographystyle{splncs03}

\end{document}
